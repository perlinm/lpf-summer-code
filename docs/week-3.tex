\documentclass[10pt,twocolumn]{article}

%%% Standard
\usepackage[margin=1in]{geometry} % One inch margins
\frenchspacing % No double spaces after periods.  Like this.
\usepackage{fancyhdr} % Easy to manage headers and footers
\usepackage{hyperref} % For linking references
\pagestyle{fancyplain} % Formatting things
\linespread{1} % Single spaced
\setlength{\parindent}{0cm} % Don't indent new paragraphs
\newcommand{\psl}{6pt} % For consistency (\psl used again below)
\parskip \psl % Place a space between paragraphs instead
\usepackage{comment} % Adds \comment{} environment
\usepackage{lastpage} % For referencing last page
\usepackage{enumitem} % Include for \setlist{}
\setlist{nolistsep} % More compact spacing between environments
\setlist[itemize]{leftmargin=*} % Nice margins for itemize
\setlist[enumerate]{leftmargin=*} % and enumerate environments

%%% One column header:
\newcommand{\head}[1]{
\twocolumn[\begin{@twocolumnfalse}
\vspace{-5mm}
\begin{center} #1 \end{center}
\end{@twocolumnfalse}]}

%%% Figures:
\usepackage{graphicx,grffile,float,subcaption} % Floats, etc.
\usepackage{multirow} % For multirow entries in tables

%%% Sections:
\usepackage{sectsty,titlesec} % Section options
\sectionfont{\large} % Define section font size
\subsectionfont{\normalsize} % Subsection font size

%%% Standard math:
\usepackage{amsmath,amssymb,amsfonts,amsthm,mathtools} % Math packages
\newcommand{\st}{\displaystyle} % For making inline math bigger
\renewcommand{\t}{\text} % For text in math environment
\renewcommand{\c}{\cdot} % Multiplication dot in math
\newcommand{\f}[2]{\dfrac{#1}{#2}} % Shortcut for fractions
\newcommand{\p}[1]{\left(#1\right)} % Parenthesis
\renewcommand{\sp}[1]{\left[#1\right]} % Square parenthesis
\newcommand{\set}[1]{\left\{#1\right\}} % Curly parenthesis
\newcommand{\R}{\mathbb{R}} % Commonly used sets
\newcommand{\Z}{\mathbb{Z}}
\newcommand{\N}{\mathbb{N}}
\renewcommand{\P}{\mathbb{P}}
\renewcommand{\S}{\mathbb{S}}
\newcommand{\fa}{\forall} % Math symbol
\newcommand{\m}[1]{\begin{pmatrix}#1\end{pmatrix}} % Matrix
\newcommand{\vm}[1]{\begin{vmatrix}#1\end{vmatrix}} % Determinant
\DeclarePairedDelimiter{\ceil}{\lceil}{\rceil}
\DeclarePairedDelimiter{\floor}{\lfloor}{\rfloor}

%%% Physics symbols, notations, operators
\usepackage[boldvectors,braket]{physymb} % Physics packages
\usepackage{tensor}
\renewcommand{\v}{\vec} % Bold vectors
\newcommand{\uv}[1]{\hat{\vec{#1}}} % Unit vectors
\newcommand{\del}{\v\nabla} % Del operator
\renewcommand{\d}{\partial} % Partial d
\newcommand{\x}{\times} % Multiplication 'x'
\renewcommand{\braket}{\Braket} % Resize brakets automatically
\newcommand{\bk}{\braket} % Shorthand
\renewcommand{\bra}{\Bra}
\renewcommand{\ket}{\Ket}
\newcommand{\ind}{\indices} % Shorthand for tensor indices
\let\vepsilon\epsilon % Remap normal epsilon to vepsilon
\let\vphi\phi % Remap normal phi to vphi
\renewcommand{\epsilon}{\varepsilon} % Prettier epsilon
\renewcommand{\phi}{\varphi} % Prettier phi
\renewcommand{\l}{\ell} % Prettier l
\renewcommand{\r}{\scriptr} % Script r

\renewcommand{\headrulewidth}{0.5pt} % Horizontal line in header
\cfoot{\thepage~of \pageref{LastPage}} % "X of Y" page labeling
\lhead{Michael A. Perlin}
\rhead{2014--07--03}

\renewcommand{\sc}{_{\t{sc}}}
\newcommand{\tm}{_{\t{tm}}}
\newcommand{\ti}{^{\p{i}}}

\begin{document}
\titlespacing{\section}{0pt}{\psl}{0pt}
\titlespacing{\subsection}{5mm}{\psl}{0pt}

\head{ \large \bf LPF Internship Progress Report (Week 3 of 10) }

\section*{Introduction}

This report explains work done in an attempt to extract information
about the space craft and thrusters on the LPF by analyzing simulated
thruster command signals and space craft acceleration data (both
linear and angular) relative to two free-falling test masses. In
simulations, eight thrusters are commanded to generate thrust
oscillating at distinct frequencies; these frequencies are used to
identify a particular thruster's contribution to the motion of the
space craft relative to the test masses. The space craft's response to
each thruster is then used to determine thruster time delays (the time
difference between a thruster command and its response), calibration
(the ratio of measured to commanded thrust) and orientation (the
altitude and azimuth in which the thruster points, relative to the
mechanical frame of the space craft). Orientation calculations assume
knowledge about the positions of the test masses and the space craft's
center of mass relative to the mechanical frame of the space craft,
and calibration calculations assume knowledge about the mass of the
space craft.

\section*{Isolating Signals}

Each thruster (indexed by $i$) is commanded to generate a thrust
$T\ti$ given by
\begin{align}
  T\ti=F\ti\sin\p{\omega\ti t+\phi\ti}+A_{\t c}\ti,
\end{align}
where $F\ti$ is a time-independent amplitude, $\omega\ti$ is an
angular frequency unique to each thruster, $\phi\ti$ is some signal
phase, and $A_{\t c}\ti$ is the sum of all commands sent to the
thruster by the space craft's attitude control system (which are
generally aperiodic). Given their total commanded signal, each
thruster's characteristic angular frequency $\omega\ti$ and amplitude
$F\ti$ are identified from the signal's amplitude spectrum. The
amplitudes for all thrusters in all simulations thus analyzed are
about $500~\t{nN}$ at frequencies $\sim10~\t{mHz}$ (with bandwidths of
$\sim100~\mu\t{Hz}$).

After identifying thruster signal frequencies, all time-series
acceleration data is filtered through a low-pass filter at a cutoff
frequency of $30~\t{mHz}$ to mitigate aliasing from high frequency
noise, downsampled from $10~\t{Hz}$ to $100~\t{mHz}$, and filtered
through band-pass filters at the characteristic frequencies of the
thrusters. Downsampling prior to using the band-pass filter reduces
the dynamic range on filter coefficients, and hence reduces numerical
errors in the filter. The resulting signals for each data set from
each band-pass filter are the isolated response of each measured
quantity, and are stored independently.

Thruster command signals are processed in the same manner (by low-pass
filtering, downsampling, and band-pass filtering each thruster at its
own frequency) to account for amplitude changes and phase shifts due
to filtering.

\section*{Time Delay}

A transfer function of the post-processing thruster signals to
accelerations is calculated to find their relative phase difference
$\Delta\phi\in[0,\pi]$ at the frequency of the thruster signal. The
direction of acceleration in response to a positive thruster signal
($+/-$) is given by whether the magnitude $\abs{\Delta\phi}$ is less
($+$) or greater ($-$) than $\pi/2$. The phase
\begin{align}
  \Delta\phi'=\left\{
    \begin{array}{ll}
      \Delta\phi & \Delta\phi \le \pi/2 \\
      \pi-\Delta\phi & \Delta\phi > \pi/2
    \end{array}\right.
\end{align}
is then taken as the direction-independent phase difference in the
signals, and used to find the time delay $\Delta t$ of the thrusters
by the relation
\begin{align}
  \Delta\phi'=\omega\Delta t\implies\Delta t=\f{\Delta\phi'}\omega.
\end{align}
As each acceleration data set yields its own delay for each thruster,
delays from all data sets at each thruster frequency are averaged to
find the delays of individual thrusters. Delays vary by simulation,
but are typically in the range $0.3$--$1$ s.

\section*{Orientation and Calibration}

Finding the calibration and orientation of the thrusters requires
first finding the contribution to linear acceleration $\v a\ti\sc$ of
the space craft by the thruster indexed by $i$, which is given by
\begin{align}
  \v a\ti\sc=\v a\ti\tm-\boldsymbol\alpha\ti\tm\times\v r\tm,
\end{align}
where $\v a\ti\tm$ and $\boldsymbol\alpha\ti\tm$ are the contributions
to linear and angular acceleration of the space craft by thruster $i$
and $\v r\tm$ is the position of a test mass relative to the space
craft's center of mass. Neglected in this equation are corrections for
the Coriolis and centrifugal forces in the rotating frame of the test
mass, which were found to be of orders $10^{-6}$ and $10^{-8}$
respectively smaller than the included terms. It is also assumed that
changes in $\v r\tm$ are negligibly small, so that it can be taken as
constant, and that the space craft's center of mass is constant
relative to the mechanical frame of the space craft, so that
accelerations relative to the mechanical frame (i.e. acceleration
data) are equal to accelerations relative to the center of mass.

After finding $\v a\sc\ti$, a covariance matrix $\v V\ti$ of its
components along the axes of the space craft's mechanical frame is
computed, with elements $V_{k\l}\ti$ given by:
\begin{align}
  V_{k\l}\ti=\bk{\p{a\ti_k-\bk{a\ti_k}}\p{a\ti_\l-\bk{a\ti_\l}}}.
\end{align}
Here $a\ti_n=\v a\sc\ti\c\v x_n$, where $\set{\v x_n}$ are the basis
vectors of the mechanical frame, and $\bk\chi$ denotes the expectation
value of the variable $\chi$. The eigenvectors of $\v V\ti$ define the
principal axes of $\v a\sc\ti$ in the basis of the mechanical frame,
with corresponding eigenvalues equal to the variance of $\v a\sc\ti$
along the respective principle axes. As $\v a\sc\ti$ is an
acceleration in response to the thrust provided by thruster $i$, the
eigenvector corresponding to the largest eigenvalue of $\v V\ti$ is
parallel to the direction of thrust; motion in orthogonal directions
(along the remaining eigenvectors) are due to signal noise, error, or
otherwise unaccounted for effects at the frequency of thruster
$i$. This eigenvector thus points in the direction of thrust (of
thruster $i$) up to sign; the ambiguity in sign is resolved using
knowledge about the direction of accelerations given a positive thrust
or by approximate knowledge of thruster orientations to find the
orientation of each thruster. Pitch (the angle above the $x$-$y$
plane) and azimuth (the angle in the $x$-$y$ plane, from the $x$ axis
to the $y$ axis) are thus determined to an accuracy of about 1
microradian.

Given that $\v a\sc\ti$ is sinusoidal, its amplitude $a\sc\ti$ is
given in terms of the variance $\p{\sigma\sc\ti}^2$ along its
principal axis by
\begin{align}
  a\sc\ti=\sqrt2\p{\sigma\sc\ti}^2.
\end{align}
The calibration $C\ti$ of thruster $i$ is then given in terms of the
mass $m\sc$ of the space craft by
\begin{align}
  C\ti=\f{m\sc a\sc\ti}{F\ti}.
\end{align}
These measured calibrations typically lie in the range $0.97$--$1.03$
(whereas all simulated calibrations are exactly 1).

\section*{Further Directions}

There are currently several current pursuits in this work. Firstly,
there are some unaccounted for discrepancies in the measured time
delays. Secondly, the calibrations of all thrusters are set to unity
in simulations. While $1-3\%$ relative errors in deduced calibrations
may attributed to error, these are larger than one would expect, and
from qualitative inspection of the actual numbers there appears to be
some systematic errors in the results. Thirdly, the behaviors of all
$\v a\sc\ti$ in the respective plane perpendicular to their principal
axes appear to be regular and periodic; such behavior is not expected
to occur absent of error (motion in this plane is expected to be
random and unpredictable), and is not presently understood. Finally,
the assumption that the center of mass of the space craft is
stationary relative to the mechanical frame may not be robust. It may
be possible to deduce the center of mass of the space craft from
thruster inputs and response data if the locations of the thrusters
are known in the mechanical frame. There is thus an effort to see what
information can be deduced from data about the center of mass if its
location is unknown.

\end{document}