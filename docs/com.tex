\documentclass[10pt]{article}

%%% Standard
\usepackage[margin=1in]{geometry} % One inch margins
\frenchspacing % No double spaces after periods.  Like this.
\usepackage{fancyhdr} % Easy to manage headers and footers
\usepackage{hyperref} % For linking references
\pagestyle{fancyplain} % Formatting things
\linespread{1} % Single spaced
\setlength{\parindent}{0cm} % Don't indent new paragraphs
\newcommand{\psl}{6pt} % For consistency (\psl used again below)
\parskip \psl % Place a space between paragraphs instead
\usepackage{comment} % Adds \comment{} environment
\usepackage{lastpage} % For referencing last page
\usepackage{enumitem} % Include for \setlist{}
\setlist{nolistsep} % More compact spacing between environments
\setlist[itemize]{leftmargin=*} % Nice margins for itemize
\setlist[enumerate]{leftmargin=*} % and enumerate environments

%%% One column header:
\newcommand{\head}[1]{
\twocolumn[\begin{@twocolumnfalse}
\vspace{-5mm}
\begin{center} #1 \end{center}
\end{@twocolumnfalse}]}

%%% Standard math:
\usepackage{amsmath,amssymb,amsfonts,amsthm,mathtools} % Math packages
\newcommand{\st}{\displaystyle} % For making inline math bigger
\renewcommand{\t}{\text} % For text in math environment
\renewcommand{\c}{\cdot} % Multiplication dot in math
\newcommand{\f}[2]{\dfrac{#1}{#2}} % Shortcut for fractions
\newcommand{\p}[1]{\left(#1\right)} % Parenthesis
\renewcommand{\sp}[1]{\left[#1\right]} % Square parenthesis
\newcommand{\set}[1]{\left\{#1\right\}} % Curly parenthesis
\newcommand{\R}{\mathbb{R}} % Commonly used sets
\newcommand{\Z}{\mathbb{Z}}
\newcommand{\N}{\mathbb{N}}
\renewcommand{\P}{\mathbb{P}}
\renewcommand{\S}{\mathbb{S}}
\newcommand{\fa}{\forall} % Math symbol
\newcommand{\m}[1]{\begin{pmatrix}#1\end{pmatrix}} % Matrix
\newcommand{\vm}[1]{\begin{vmatrix}#1\end{vmatrix}} % Determinant
\DeclarePairedDelimiter{\ceil}{\lceil}{\rceil}
\DeclarePairedDelimiter{\floor}{\lfloor}{\rfloor}

%%% Physics symbols, notations, operators
\usepackage[boldvectors,braket]{physymb} % Physics packages
\usepackage{tensor}
\renewcommand{\v}{\vec} % Bold vectors
\newcommand{\uv}[1]{\hat{\vec{#1}}} % Unit vectors
\newcommand{\del}{\v\nabla} % Del operator
\renewcommand{\d}{\partial} % Partial d
\newcommand{\x}{\times} % Multiplication 'x'
\renewcommand{\braket}{\Braket} % Resize brakets automatically
\newcommand{\bk}{\braket} % Shorthand
\renewcommand{\bra}{\Bra}
\renewcommand{\ket}{\Ket}
\newcommand{\ind}{\indices} % Shorthand for tensor indices
\let\vepsilon\epsilon % Remap normal epsilon to vepsilon
\let\vphi\phi % Remap normal phi to vphi
\renewcommand{\epsilon}{\varepsilon} % Prettier epsilon
\renewcommand{\phi}{\varphi} % Prettier phi
\renewcommand{\l}{\ell} % Prettier l
\renewcommand{\r}{\scriptr} % Script r

\renewcommand{\headrulewidth}{0.5pt} % Horizontal line in header
\cfoot{\thepage~of \pageref{LastPage}} % "X of Y" page labeling
\lhead{Michael A. Perlin}
\rhead{2014--07--03}

%%% Figures:
\usepackage{graphicx,grffile,float,subcaption} % Floats, etc.
\usepackage{multirow} % For multirow entries in tables

\newcommand{\vv}{\boldsymbol} % Bold vectors (greek letters)
\newcommand{\uvv}[1]{\uv{\vv{#1}}} % Bold unit vectors (greek letters)
\renewcommand{\sc}{_{\t{sc}}}
\renewcommand{\th}{_{\t{th}}}
\newcommand{\tm}{_{\t{tm}}}
\newcommand{\I}{\mathcal{I}} % script I

\begin{document}

\begin{center}
  \large \bf Solution for the Space Craft's Center of Mass
\end{center}

The purpose of this derivation is to find the center of mass of the
space craft in its mechanical frame from the following known
quantities:
\begin{itemize}
\item inertial properties of the space craft (i.e. mass and moment of
  inertia)
\item locations of test masses and thrusters (in the mechanical frame
  of the space craft)
\item accelerations of the test masses in response to the thrusters
\end{itemize}
For the purpose of this derivation, we use quantities and data for
only one test mass and one thruster. Table \ref{variables} contains
and describes all base variables used below. All positions are in the
mechanical frame of the space craft.

\begin{table}[H]
  \centering
  \caption{Variables used in derivation}
  \begin{tabular}{|c|l|l|} \hline
    Symbol & Description & Status \\ \hline
    $m\sc$ & mass of space craft & known (measured) \\
    $\I\sc$ & moment of inertia of space craft & known (measured) \\
    $\v r\tm$ & position of test mass & known (measured) \\
    $\v r\th$ & position of thruster & known (measured) \\
    $\v r\sc$ & position of space craft's center of mass & unknown \\
    $\v a\tm$ & linear acceleration of test mass (relative to space
    craft) & known (data) \\
    $\v a\sc$ & linear acceleration of space craft (absolute)
    & unknown \\
    $\vv\alpha\tm$ & angular acceleration of test mass (relative to
    space craft) & known (data) \\
    $\vv\alpha\sc$ & angular acceleration of space craft (absolute)
    & known ($=-\vv\alpha\tm$) \\
    $\v F\th$ & force exerted by thruster on space craft & unknown \\
    $\vv\tau\th$ & torque exerted by thruster on space craft &
    known ($=\I\sc\vv\alpha\sc$) \\
    \hline
  \end{tabular}
  \label{variables}
\end{table}

The torque exerted on the space craft by a thruster is given by
\begin{align}
  \vv\tau\th=\p{\v r\th-\v r\sc}\times\v F\th.
\end{align}
Substituting the space craft's mass, moment of inertia, and
accelerations yields
\begin{align}
  \I\sc\vv\alpha\sc=\p{\v r\th-\v r\sc}\times m\sc\v a\sc.
\end{align}
Neglecting the Coriolis force and centripetal acceleration in the
frame of the test mass, we can substitute measured accelerations of
the test mass to get
\begin{align}
  -\I\sc\vv\alpha\tm=\p{\v r\th-\v r\sc}\times m\sc\p{\v
    a\tm-\vv\alpha\tm\times\sp{\v r\tm-\v r\sc}}. \label{accels}
\end{align}
Defining the vector
\begin{align}
  \v u=\v a\tm-\vv\alpha\tm\times\v r\tm,
\end{align}
(\ref{accels}) can be written as
\begin{align}
  -m\sc^{-1}\I\sc\vv\alpha\tm=\p{\v r\th-\v r\sc} \times\p{\v
    u+\vv\alpha\tm\times\v r\sc}.
\end{align}
Separating known and unknown terms yields
\begin{align}
  \v r\th\times\v u+m\sc^{-1}\I\sc\vv\alpha\tm =\v r\sc\times\v u
  +\p{\v r\sc-\v r\th}\times\p{\vv\alpha\tm\times\v r\sc}
  \label{solve}
\end{align}
Assuming that the solution space $\mathcal R\sc$ of possible positions
$\v r\sc$ satisfying (\ref{solve}) is a differentiable manifold, the
equality in (\ref{solve}) does not change under a variation $\v
r\sc\to\v r\sc+\delta\uv p$ where $\delta$ is an infinitesimally small
quantity and $\uv p$ is a unit vector tangent to $\mathcal R\sc$ at
some valid $\v r\sc$. We can thus say that
\begin{align}
  \v r\sc\times\v u +\p{\v r\sc-\v r\th}\times\p{\vv\alpha\tm\times\v
    r\sc} =\p{\v r\sc+\delta\uv p}\times\v u +\p{\v r\sc+\delta\uv
    p-\v r\th}\times\p{\vv\alpha\tm\times\sp{\v r\sc+\delta\uv p}}.
  \label{variation}
\end{align}
Canceling out all terms not containing $\delta\uv p$ on both sides of
(\ref{variation}) yields:
\begin{align}
  0=\delta\uv p\times\v u+\p{\v r\sc-\v r\th}\times
  \p{\vv\alpha\tm\times\delta\uv p}+\delta\uv p\times
  \p{\vv\alpha\tm\times\sp{\v r\sc+\delta\uv p}}.
\end{align}
Collecting terms by powers of $\delta$,
\begin{align}
  0=\delta\p{\uv p\times\v u+\sp{\v r\sc-\v r\th}\times
    \sp{\vv\alpha\tm\times\uv p}+\uv p\times\sp{\vv\alpha\tm\times\v
      r\sc}}+\delta^2\p{\uv p\times\sp{\vv\alpha\tm\times\uv p}}.
  \label{deltas}
\end{align}
As terms in different powers of $\delta$ must independently satisfy
(\ref{deltas}), the appearance of the $\delta^2$ term therein implies
\begin{align}
  \uv p=\uvv\alpha\tm.
\end{align}
The terms in (\ref{deltas}) which are in the first power of $\delta$
thus satisfy
\begin{align}
  0=\uvv\alpha\tm\times\v u+\p{\v r\sc-\v r\th}\times
  \p{\vv\alpha\tm\times\uvv\alpha\tm}
  +\uvv\alpha\tm\times\p{\vv\alpha\tm\times\v
    r\sc}=\uvv\alpha\tm\times\p{\v u+\vv\alpha\tm\times\v r\sc}.
  \label{perp}
\end{align}
Removing the component in the cross product parallel to
$\uvv\alpha\tm$, we can say that
\begin{align}
  0=\uvv\alpha\tm\times\p{\v u-\sp{\v u\c\uvv\alpha\tm}\uvv\alpha\tm
    +\vv\alpha\tm\times\v r\sc}\equiv\uvv\alpha\tm\times\p{\v
    u^\perp+\vv\alpha\tm\times\v r\sc}, \label{perp}
\end{align}
where we have implicitly defined the vector
\begin{align}
  \v u^\perp\equiv\v u-\p{\v u\c\uvv\alpha\tm}\uvv\alpha\tm=\v
  a\tm-\p{\v a\tm\c\uvv\alpha\tm}\uvv\alpha\tm-\vv\alpha\tm\times\v
  r\tm.
\end{align}
As all of $\v u^\perp+\vv\alpha\tm\times\v r\sc$ is perpendicular to
$\uvv\alpha\tm$, to satisfy (\ref{perp}) we must have
\begin{align}
  \v r\sc\times\vv\alpha\tm=\v u^\perp. \label{result}
\end{align}
While this result does not allow us to solve for $\v r\sc$ explicitly,
we can find the component
\begin{align}
  \v r\sc^\perp=\v r\sc-\p{\v r\sc\c\uvv\alpha\tm}\uvv\alpha\tm
\end{align}
perpendicular to $\uvv\alpha\tm$, which is given by:
\begin{align}
  \v r\sc^\perp=\f{\abs{\v u^\perp}}{\abs{\vv\alpha\tm}}
  \uvv\alpha\tm\times\uv u^\perp.
\end{align}
We thus conclude that
\begin{align}
  \mathcal R\sc=\v r\sc^\perp+\R\uvv\alpha\tm.
\end{align}
Though one thruster does not provide enough information to determine
$\v r\sc$, using data from multiple thrusters one can find the
intersection of all $\mathcal R\sc$ (or, in practice, the point of
closest approach to all $\mathcal R\sc$) to get the actual position
$\v r\sc$.

\end{document}